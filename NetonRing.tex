%%%%%%%%%%%%%%%%%%%%%核心代码来自http://tex.stackexchange.com/a/159234
\documentclass{standalone}%[varwidth=18cm,] mutlti=true [tikz]
\RequirePackage{luatex85}
\def\pgfsysdriver{pgfsys-pdftex.def}
\usepackage{luatexja-fontspec}
\usepackage{tikz}
\usetikzlibrary{shapes,
                calc,
                fpu,
                patterns,
                decorations.pathreplacing,
                arrows,
                shapes.geometric,
                decorations.pathmorphing,
                spy,
                backgrounds,
                shapes.callouts,
                decorations.fractals,
                decorations.markings,
                }
\usepackage{pgfplots}
    \usepgfplotslibrary{fillbetween}
\usepackage{pdftexcmds}
\usepackage{xparse}
\usepackage{etoolbox}
\usepackage{animate}
\usepackage{amsmath}
\usepackage{ifthen}
\usepackage{tagging}
        %\usetag{Draft} %显示所有标记 变量
\usepackage{fp}
\usepackage{graphicx}
        \DeclareGraphicsExtensions{.pdf,.png,.jpg,.mps}
%\usepackage{lua-visual-debug}


%%%%%%%%%%%%%%%%%%%%%%%%%%%%%%%%%%%%%%%%%%%%%%%%%%%%%%%%%%%%%%%%%%%%%%%%%%%%%%%
%%%%%%%%%%%%%%%%%%%%%%%%%%%%%%%%%%%%%%%%%%%%%%%%%%%%%%%%%%%%%%%%%%%%%%%%%%%%%%%
%%%%%%%%%%%%%%%%%%%%%%%%%%%%%%%%%%%%%%%%%%%%%%%%%%%%%%%%%%%%%%%%%%%%%%%%%%%%%%%
%%%%%%%%%%%%%%%%%%%%%%%%%%%%%%%%%%%%%%%%%%%%%%%%%%%%%%%%%%%%%%%%%%%%%%%%%%%%%%%
%载入命令 设置
\makeatletter\input{NRConfig/Font.set}\makeatother %载入字体设置
\makeatletter\input{NRConfig/TheBBCstyleCUG.set}\makeatother %BBC LOGO 形式 CUG LOGO
\makeatletter\input{NRConfig/TikzVariables.tikzconfig}\makeatother %载入全局变量
\makeatletter\input{NRConfig/FillPattern.tikzconfig}\makeatother %填充效果
\makeatletter\input{NRConfig/HelpLines.tikzset}\makeatother %坐标辅助线
\makeatletter\input{NRConfig/LaserBeamStyle.tikzset}\makeatother %线两边模糊效果
\makeatletter\input{NRConfig/CoteFunction.tikzcmd}\makeatother      %CAD尺寸标记效果
\makeatletter\input{NRConfig/DrawNewtonRing.tikzcmd}\makeatother    %画完整牛顿环0-\MaxRings环,并标识出每环水平方向左右两边 及左上 右上的标记点
\makeatletter\input{NRConfig/IndicateRingNumber.tikzcmd}\makeatother %显示怎么数环
\makeatletter\input{NRConfig/DmDsLmLsCote.tikzcmd}\makeatother      %标记Dm Ds Lm Ls 和它们之间的关系
\makeatletter\input{NRConfig/DrawLenGlass.tikzcmd}\makeatother      %画产生牛顿环的实物示意图
\makeatletter\input{NRConfig/DrawCross.tikzcmd}\makeatother         %画目镜十字线命令使用了线两边模糊效果,因此LaserBeamStyle.tikzset应先加载
\makeatletter\input{NRConfig/ShowHowToTakeReading.tikzcmd}\makeatother %演示如何读数 m=\m,s=\s
\makeatletter\input{NRConfig/TikzPictureStyleDefine.tikzset}\makeatother %预定义的形式

%%%%%%%%%%%%%%%%%%%%%%%%%%%%%%%%%%%%%%%%%%%%%%%%%%%%%%%%%%%%%%%%%%%%%%%%%%%%%%%
%%%%%%%%%%%%%%%%%%%%%%%%%%%%%%%%%%%%%%%%%%%%%%%%%%%%%%%%%%%%%%%%%%%%%%%%%%%%%%%
%%%%%%%%%%%%%%%%%%%%%%%%%%%%%%%%%%%%%%%%%%%%%%%%%%%%%%%%%%%%%%%%%%%%%%%%%%%%%%%
%%%%%%%%%%%%%%%%%%%%%%%%%%%%%%%%%%%%%%%%%%%%%%%%%%%%%%%%%%%%%%%%%%%%%%%%%%%%%%%
\usetag{0}%%%%%只有玻璃和透镜 光路
%\usetag{1}%%%%装置 和牛顿环的像
%\usetag{2}%%%%%%%仅装置但包含辅助线
%\usetag{3}%%%%%怎么数环            %pdf动画单独一页standalone 选项为不要tikz
    %\usetag{3ToSWF}%pdf动画每帧为一页单独输出  此时 standalone 选项必须加 tikz ,要转换swf文件时才使用此行
%\usetag{4}%%%%dm ds lm ls 关系
%\usetag{5}%怎么读数              %pdf动画单独一页standalone 选项为不要tikz
    %\usetag{5ToSWF}%pdf动画每帧为一页单独输出    此时 standalone 选项必须加 tikz ,要转换swf文件时才使用此行 
%%%%%%%%%%%%%%%%%%%%%%%%%%%%%%%%%%%%%%%%%%%%%%%%%%%%%%%%%%%%%%%%%%%%%%%%%%%%%%%
%%%%%%%%%%%%%%%%%%%%%%%%%%%%%%%%%%%%%%%%%%%%%%%%%%%%%%%%%%%%%%%%%%%%%%%%%%%%%%%
%%%%%%%%%%%%%%%%%%%%%%%%%%%%%%%%%%%%%%%%%%%%%%%%%%%%%%%%%%%%%%%%%%%%%%%%%%%%%%%
%%%%%%%%%%%%%%%%%%%%%%%%%%%%%%%%%%%%%%%%%%%%%%%%%%%%%%%%%%%%%%%%%%%%%%%%%%%%%%%


\newcommand{\Logo}{\scalebox{0.05}{\includegraphics{NRConfig/CUGLogo}}}%校徽


\begin{document}%
%
\pagecolor{yellow!50}%页面颜色 50%黄色
%%%%%%%%%%%%%%%%%%%%%%%%%%%%%%%%%%%%%%%%%%%%%%%%%%%%%%%%%%%%%%%%%%%%%%%%%%%%%%%%%%%%%%%%%%%%%%%%%%%%%%%%%%%%%%%%%%%%%%%%%%%%
%%%%%%%%%%%%%%%%%%%%%%%%%%%%%%%%%%%%%%%%%%%%%%%%%%%%%%%%%%%%%%%%%%%%%%%%%%%%%%%%%%%%%%%%%%%%%%%%%%%%%%%%%%%%%%%%%%%%%%%%%%%%
%%%%%%%%%%%%%%%%%%%%%%%%%%%%%%%%%%%%%%%%%%%%%%%%%%%%%%%%%%%%%%%%%%%%%%%%%%%%%%%%%%%%%%%%%%%%%%%%%%%%%%%%%%%%%%%%%%%%%%%%%%%%
\tagged{0}{%只有玻璃和透镜
            \begin{tikzpicture}%
            \path[use as bounding box] (-6.2,-5.8) rectangle (6.2,4.85); %只显示该范围内图形
            %\draw (-8,-8) to[grid with coordinates] (7,6); %坐标 网格辅助线
            %%\DrawAFullNewtonRingsNormal
            \DrawRingOnlyCoordinate%
            %%\DrawLenGlassWithRings
            \node at ($(0,-1.75*\CanvasH)+(CoorL\MaxRings|-CoorUU\MaxRings)$)[opacity=0.62,anchor=center] (Logo){\Logo};%
            \node at (Logo)[yshift=20,scale=0.5]{\TheCUG};%
            \DrawLenGlassWithoutAuxiliaryLine%
            \end{tikzpicture}%
}%
%%%%%%%%%%%%%%%%%%%%%%%%%%%%%%%%%%%%%%%%%%%%%%%%%%%%%%%%%%%%%%%%%%%%%%%%%%%%%%%%%%%%%%%%%%%%%%%%%%%%%%%%%%%%%%%%%%%%%%%%%%%%
%%%%%%%%%%%%%%%%%%%%%%%%%%%%%%%%%%%%%%%%%%%%%%%%%%%%%%%%%%%%%%%%%%%%%%%%%%%%%%%%%%%%%%%%%%%%%%%%%%%%%%%%%%%%%%%%%%%%%%%%%%%%
%%%%%%%%%%%%%%%%%%%%%%%%%%%%%%%%%%%%%%%%%%%%%%%%%%%%%%%%%%%%%%%%%%%%%%%%%%%%%%%%%%%%%%%%%%%%%%%%%%%%%%%%%%%%%%%%%%%%%%%%%%%%
%%%%%%%%%%%%%%%%%%%%%%%%%%%%%%%%%%%%%%%%%%%%%%%%%%%%%%%%%%%%%%%%%%%%%%%%%%%%%%%%%%%%%%%%%%%%%%%%%%%%%%%%%%%%%%%%%%%%%%%%%%%%
\tagged{1}{%
            %%%%%%%%%%%%%%%%%%%%%%%%%%%%牛顿环和装置
            \begin{tikzpicture}%
            \path[use as bounding box] (-6.0,-5.45) rectangle (6.0,21.45); %只显示该范围内图形
            %\draw (-7,-5.6) to[grid with coordinates] (7,22); %坐标 网格辅助线
            \DrawAFullNewtonRingsNormal%
            \node at ($(0.15*\CanvasW,17.76*\CanvasH)+(CoorL\MaxRings|-CoorUU\MaxRings)$)[opacity=0.62,anchor=center](Logo) {\Logo};%
            \node at (Logo)[yshift=20,scale=0.5]{\TheCUG};%
            \DrawLenGlassWithRings%
            \PutAnEye%
            \end{tikzpicture}%
}%
%%%%%%%%%%%%%%%%%%%%%%%%%%%%%%%%%%%%%%%%%%%%%%%%%%%%%%%%%%%%%%%%%%%%%%%%%%%%%%%%%%%%%%%%%%%%%%%%%%%%%%%%%%%%%%%%%%%%%%%%%%%%
%%%%%%%%%%%%%%%%%%%%%%%%%%%%%%%%%%%%%%%%%%%%%%%%%%%%%%%%%%%%%%%%%%%%%%%%%%%%%%%%%%%%%%%%%%%%%%%%%%%%%%%%%%%%%%%%%%%%%%%%%%%%
%%%%%%%%%%%%%%%%%%%%%%%%%%%%%%%%%%%%%%%%%%%%%%%%%%%%%%%%%%%%%%%%%%%%%%%%%%%%%%%%%%%%%%%%%%%%%%%%%%%%%%%%%%%%%%%%%%%%%%%%%%%%
\tagged{2}{%
            %%%%%%%%%%%%%%%%%%%%%%%%%%%%%%装置和辅助线
            \begin{tikzpicture}%
            \path[use as bounding box] (-6.0,-5.62) rectangle (6.0,10.29); %只显示该范围内图形
            %\draw (-7,-5.6) to[grid with coordinates] (7,10.25); %坐标 网格辅助线
            \DrawRingOnlyCoordinate%
            \node at ($(0.46*\CanvasW,-0.95*\CanvasH)+(CoorL\MaxRings|-CoorUU\MaxRings)$)[xshift=-18,yshift=135,opacity=0.62,anchor=center] (Logo){\Logo};%
            \node at (Logo)[yshift=20,scale=0.5]{\TheCUG};%
            \DrawLenGlassOnlyWithAuxiliaryLine%
            \end{tikzpicture}%
}%
%
%
%%%%%%%%%%%%%%%%%%%%%%%%%%%%%%%%%%%%%%%%%%%%%%%%%%%%%%%%%%%%%%%%%%%%%%%%%%%%%%%%%%%%%%%%%%%%%%%%%%%%%%%%%%%%%%%%%%%%%%%%%%%%
%%%%%%%%%%%%%%%%%%%%%%%%%%%%%%%%%%%%%%%%%%%%%%%%%%%%%%%%%%%%%%%%%%%%%%%%%%%%%%%%%%%%%%%%%%%%%%%%%%%%%%%%%%%%%%%%%%%%%%%%%%%%
%%%%%%%%%%%%%%%%%%%%%%%%%%%%%%%%%%%%%%%%%%%%%%%%%%%%%%%%%%%%%%%%%%%%%%%%%%%%%%%%%%%%%%%%%%%%%%%%%%%%%%%%%%%%%%%%%%%%%%%%%%%%
\pgfmathsetmacro{\IndicateNumberTotalFrame}{int(\PointOutRingMax+2)}%演示怎么数环时总帧数
\tagged{3}{%
%%%%%%%%%%%%%%%%%%%%%%%%%%%怎么数环
                                                %pdf动画
            \begin{animateinline}[controls,%
                                  %autoplay,%
                                  %loop,%
                                  buttonsize=1.8em,%
                                  %poster=1,%
                                  method=ocg,%
                                  label=RingsRead,%
                                  ]{0.75}%
            \multiframe{\IndicateNumberTotalFrame}{n=1+1}{%\IndicateNumberTotalFrame
                                                            \begin{tikzpicture}[]%
                                                            \path[use as bounding box] (-5.45,-5.8) rectangle (5.45,5.45); %只显示该范围内图形
                                                            %\draw (-6.2,-5.8) to[grid with coordinates] (6.2,5.8); %坐标 网格辅助线
                                                            \DrawAFullNewtonRingsNormal%
                                                            \node at ($(0.42*\CanvasW,-1.06*\CanvasH)+(CoorL\MaxRings|-CoorUU\MaxRings)$)[opacity=0.62,anchor=center] (Logo){\Logo};%
                                                            \node at (Logo)[yshift=20,scale=0.5]{\TheCUG};
                                                            \IndicateRingNumber%
                                                            \end{tikzpicture}%
                                                        }%
            \end{animateinline}%
}%
%%%%%%%%%%%%%%%%%%%%%%%%%%%%%%%%%%%%%%%%%%%%%%%%%%%%%%%%%%%%%%%%%%%%%%%%%%%%%%%%%%%%%%%%%%%%%%%%%%%%%%%%%%%%%%%%%%%%%%%%%%%%
\tagged{3ToSWF}{%
%%%%%%%%%%%%%%%%%%%%%%%%%%%怎么数环
                                                %swf动画
            \foreach \i in {1,...,\IndicateNumberTotalFrame}{%\IndicateNumberTotalFrame
                                                            \begin{tikzpicture}[]%
                                                            \path[use as bounding box] (-5.45,-6.72) rectangle (5.45,5.5); %只显示该范围内图形
                                                            %\draw (-6.2,-5.8) to[grid with coordinates] (6.2,5.8); %坐标 网格辅助线
                                                            \DrawAFullNewtonRingsNormal%
                                                            \node at ($(0.42*\CanvasW,-\CanvasH)+(CoorL\MaxRings|-CoorUU\MaxRings)$)[opacity=0.62,anchor=center] (Logo){\Logo};%
                                                            \node at (Logo)[yshift=20,scale=0.5]{\TheCUG};%
                                                            \IndicateRingNumber%
                                                            \end{tikzpicture}%
                                                            }%
}%
%
%%%%%%%%%%%%%%%%%%%%%%%%%%%%%%%%%%%%%%%%%%%%%%%%%%%%%%%%%%%%%%%%%%%%%%%%%%%%%%%%%%%%%%%%%%%%%%%%%%%%%%%%%%%%%%%%%%%%%%%%%%%%
%%%%%%%%%%%%%%%%%%%%%%%%%%%%%%%%%%%%%%%%%%%%%%%%%%%%%%%%%%%%%%%%%%%%%%%%%%%%%%%%%%%%%%%%%%%%%%%%%%%%%%%%%%%%%%%%%%%%%%%%%%%%
%%%%%%%%%%%%%%%%%%%%%%%%%%%%%%%%%%%%%%%%%%%%%%%%%%%%%%%%%%%%%%%%%%%%%%%%%%%%%%%%%%%%%%%%%%%%%%%%%%%%%%%%%%%%%%%%%%%%%%%%%%%%
\tagged{4}{%
%%%%%%%%%%%%%%%%%%%%%%%%%%dm ds lm ls 关系
            \begin{tikzpicture}%
            \path[use as bounding box] (-6.55,-6.4) rectangle (5.8,6.48); %只显示该范围内图形
            %\draw (-7.0,-6.76) to[grid with coordinates] (5.4,6.48); %坐标 网格辅助线
            \DrawRingMAndS%
            \node at ($(-0.15*\CanvasW,0)+(CoorL\MaxRings|-CoorUU\MaxRings)$) [yshift=4,opacity=0.62,anchor=center] (Logo) {\Logo};%
            \node at (Logo)[yshift=20,scale=0.5]{\TheCUG};%
            \DmDsCote%
            \LmLsCote%
            \DmDsLmLsRelationship%
            \end{tikzpicture}%
}%
%
%%%%%%%%%%%%%%%%%%%%%%%%%%%%%%%%%%%%%%%%%%%%%%%%%%%%%%%%%%%%%%%%%%%%%%%%%%%%%%%%%%%%%%%%%%%%%%%%%%%%%%%%%%%%%%%%%%%%%%%%%%%%
%%%%%%%%%%%%%%%%%%%%%%%%%%%%%%%%%%%%%%%%%%%%%%%%%%%%%%%%%%%%%%%%%%%%%%%%%%%%%%%%%%%%%%%%%%%%%%%%%%%%%%%%%%%%%%%%%%%%%%%%%%%%
%%%%%%%%%%%%%%%%%%%%%%%%%%%%%%%%%%%%%%%%%%%%%%%%%%%%%%%%%%%%%%%%%%%%%%%%%%%%%%%%%%%%%%%%%%%%%%%%%%%%%%%%%%%%%%%%%%%%%%%%%%%%
\pgfmathsetmacro{\TakeReadingTotalFrame}{int(3*(\m+2)+1)}%演示怎么读数时总帧数
\tagged{5}{%
%%%%%%%%%%%%%%%%%%%%%%%%%%%怎么测量dm ds
                                               %pdf动画
            \begin{animateinline}[controls,%
                                  %autoplay,
                                  %loop,
                                  buttonsize=1.8em,%
                                  %poster=1,%
                                  method=ocg,%
                                  label=TakeReading,%
                                  ]{1}%
            \multiframe{\TakeReadingTotalFrame}{N=1+1}{%73 %3 poster=1   \multiframe{\TakeReadingTotalFrame}{N=1+1}
                                                \begin{tikzpicture}[spy using outlines={circle, magnification=6, connect spies,},]%%dash pattern= on 1 off 1
                                                %\path[use as bounding box] (-6,-5.4) rectangle (6,6.48); %只显示该范围内图形
                                                %\draw (-8,-8) to[grid with coordinates] (8,8); %坐标 网格辅助线
                                                %\DrawAFullNewtonRingsShade%
                                                \DrawAFullNewtonRingsNormal
                                                %\DrawRingMAndS
                                                \node at ($(-0.63*\CanvasW,0)+(CoorL\MaxRings|-CoorUU\MaxRings)$) [opacity=0.62,anchor=center] (Logo) {\Logo};
                                                \node at (Logo)[yshift=20,scale=0.5]{\TheCUG};
                                                \ShowHowToTakeReading
                                                \end{tikzpicture}%
                                                     }%fram结束
            \end{animateinline}%
}%
%%%%%%%%%%%%%%%%%%%%%%%%%%%%%%%%%%%%%%%%%%%%%%%%%%%%%%%%%%%%%%%%%%%%%%%%%%%%%%%%%%%%%%%%%%%%%%%%%%%%%%%%%%%%%%%%%%%%%%%%%%%%
\tagged{5ToSWF}{%
%%%%%%%%%%%%%%%%%%%%%%%%%%%怎么测量dm ds
                                        %swf动画
            \foreach \i in {1,...,\TakeReadingTotalFrame}{% 1-\TakeReadingTotalFrame
                                                            \begin{tikzpicture}[spy using outlines={circle, magnification=6, connect spies},]%dash pattern= on 1 off 1%
                                                            %\DrawAFullNewtonRingsShade%
                                                            \DrawAFullNewtonRingsNormal
                                                            \node at ($(1.22*\CanvasW,-0.96*\CanvasH)-(CoorLU\MaxRings|-CoorUU\MaxRings)$) {\tagged{Draft}{\tikz\draw[fill=red,color=red] circle(2pt);}{}};%扩大画布放置播放按钮
                                                            \node at ($(-0.63*\CanvasW,0)+(CoorL\MaxRings|-CoorUU\MaxRings)$) [opacity=0.62,anchor=center] (Logo) {\Logo};%放置Logo
                                                            \node at (Logo)[yshift=20,scale=0.5]{\TheCUG};%放置CUG字母
                                                            \ShowHowToTakeReading%
                                                            \end{tikzpicture}%
                                                            }%
}%
%
\end{document} 